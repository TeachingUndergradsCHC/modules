\documentclass{article}

\usepackage{fullpage}
\usepackage{url}

\begin{document}

In this lab, you will be implementing some simple image processing
commands in SYCL.
Begin by opening the Colab notebook:

{\footnotesize \url{https://github.com/TeachingUndergradsCHC/modules/blob/master/Programming/sycl/syclBlur.ipynb}}

Follow the directions in that notebook to set up SYCL in your Colab
instance.
One of the directions asks you to upload the files that we use this lab.
The files are available from the module repo:

{\footnotesize
  \url{https://github.com/TeachingUndergradsCHC/modules/blob/master/Programming/sycl/}}

You'll need to upload the library code that manages images in the .bmp
file format ({\tt ppmFile.h} and {\tt ppmFile.c}) and also the image
that you want to process; the sample I provide is {\tt 640x426.bmp}.

The second to last code cell in the notebook runs the ``Hello World''
program discussed in class.
Run this example to check that everything is working.
It should print out a series of messages, one per thread.

Once you've verified that SYCL is installed and running properly, it's
time to begin the main part of lab.
The last code cell in the notebook is the first version of our image
processing program.
This version removes all the red from
our sample image (640x426.bmp) and creates a new file
{\tt out.bmp}.
(You may have to refresh the file panel to the left by hitting the
button with circular arrow to see it.)
Download this file and observe that it looks basically the same as our
original image, but the colors are slightly shifted.
The image is stored as an array of RGB values, each color in a
separate {\em channel} of the pixel.

Let's look through the kernel to see how it works; it's the function
{\tt cgh}.
This kernel is called one each pixel of the input image.
First, the variable {\tt idx} is used to index blocks of pixels,
each pixel having 3 colors. The invocations are organized into 
independent computations for each pixel.

The image itself is stored linearly as triples of integers for each
pixel.
The elements of these pixels are called the {\em red channel}, the
{\em green channel}, and the {\em blue channel} respectively because
they store the amount of red, green, and blue in their pixel.
The given code simply changes the value in the red channel to 0,
leaving the other channels unchanged.
This has the effect of removing all the red from the image.

As a first task, let's change this kernel to convert the image into 
grayscale (black and white).
To do this, take the values of the red, green, and blue channels and
average them (add them up and divide by 3).
Set the value of each channel to be this average.
Do this and make sure it's working before proceeding.

Next, you will modify the kernet to create a blur effect.
This is also done by taking an average, but a different kind.
Each channel is blurred separately--- each gets the average of that
channel's value for nearby pixels.
Change the kernel to take an integer argument {\tt k} and 
call it with value 5.
This argument gives the distance over which you want to take the
average.
Specifically, compute the average of each channel value over all
pixels differing by at most {\tt k} in each coordinate.
The pixels in this region will form a square with side length
{\tt 2k+1} since each coordinate differs by a value between -5 and 5.
(Formally, the pixels in this region are those whose $L_\infty$
 distance is $\leq$ {\tt k}.
 It's ok if those words don't make sense to you (yet).)

\end{document}
