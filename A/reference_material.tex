\documentclass[11pt]{article}

\usepackage{url}
\usepackage{etoolbox}
\patchcmd{\thebibliography}{\section*{\refname}}{}{}{}

\newenvironment{list2}{
  \begin{list}{$\bullet$}{%
      \setlength{\itemsep}{0in}
      \setlength{\parsep}{0in} \setlength{\parskip}{0in}
      \setlength{\topsep}{0in} \setlength{\partopsep}{0in} 
      \setlength{\leftmargin}{0.5in}}}{\end{list}}\newenvironment{list1}{
  \begin{list}{\ding{113}}{%
      \setlength{\itemsep}{0in}
      \setlength{\parsep}{0in} \setlength{\parskip}{0in}
      \setlength{\topsep}{0in} \setlength{\partopsep}{0in} 
      \setlength{\leftmargin}{0.17in}}}{\end{list}}


\def\fullline{		% hrules only listen to \hoffset
  \nointerlineskip	% so I have this code	  
  \moveleft\hoffset\vbox{\hrule width\textwidth} 
  \nointerlineskip
}

\renewcommand{\rmdefault}{phv} % arial, uncomment to use times new roman
\renewcommand{\sfdefault}{phv} % arail, uncomment to use times new roman
\oddsidemargin=0.0in
\textwidth=6.0in

\begin{document}

\section*{Module A: Elementary Notions of Heterogeneous Computing}
\section*{Reference Material}

There is no single resource that can serve as a good reference for all of the concepts covered in
this module. Furthermore, as yet, there is no definitive text on heterogeneous computing. Below we
list a book and several articles that covers fundamentals of heterogeneous computing and the
performance and programmability issues thereof. The list includes two survey articles written for an
audience with some background in heterogeneous computing. We also include two papers that
specifically deal with the pedagogy of heterogeneous computing.  

\nocite{Hwu:Book15}
\nocite{Zahran:CACM17}
\nocite{Mittal:CSUR15}
\nocite{Mittal:CSUR16}
\nocite{Gopalakrishnan:IPDPSW12}
\nocite{Gutierrez:EDUHPC18}

\bibliography{edu}
\bibliographystyle{unsrt}
\end{document}       
