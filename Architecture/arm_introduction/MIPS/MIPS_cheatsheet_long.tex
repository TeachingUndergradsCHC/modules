\documentclass{article}

\usepackage{fullpage}
\usepackage{verbatim}
\usepackage{alltt}

%indicate that it's a pseudoinstruction
\newcommand{\pseudo}{{\ensuremath\ast}}

\begin{document}

\begin{center}
  \bf\LARGE Assembly Operations
\end{center}

The right column gives the command followed by its arguments.
Arguments beginning with $r$ are registers.
An {\tt imm} argument is a number; its subscript
is the maximum size in bits.
For commands with a result, {\tt rd} is the
destination.
Operations marked with \pseudo\ are pseudoinstructions, which
translate into other instructions.

\vspace{1em}
\noindent
\textbf{Arithmetic operations}
\begin{center}
\begin{tabular}{ll}
Add & {\tt add rd, rs, rt} \\
%Add w/o overflow & {\tt addu rd, rs, rt} \\
Add immediate & {\tt addi rd, rs, }$\mathtt{imm}_{16}$ \\
%Add immediate w/o overflow & {\tt addiu rd, rs, }$\mathtt{imm}_{16}$ \\
Subtract & {\tt sub rd, rs, rt} \\
%Subtract w/o overflow & {\tt subu rd, rs, rt} \\
Multiply\pseudo & {\tt mulo rd, rs, rt} \\
%Multiply w/o overflow & {\tt mul rd, rs, rt} \\
Unsigned multiply\pseudo & {\tt mulou rd, rs, rt} \\
Divide $s$ by $t$\pseudo & {\tt div rd, rs, rt} \\
%Divide w/o overflow & {\tt divu rd, rs, rt} \\
Remainder\pseudo & {\tt rem rd, rs, rt} \\
Unsigned remainder\pseudo & {\tt remu rd, rs, rt} \\
Absolute value & {\tt abs rd, rs} \\
Negate value\pseudo & {\tt neg rd, rs} \\
%Negate value w/o overflow\pseudo & {\tt negu rd, rs}
\end{tabular}
\end{center}

\vspace*{-1em}
\noindent
\textbf{Bit operations}
\begin{center}
\begin{tabular}{ll}
Count leading ones & {\tt clo rd, rs} \\
Count leading zeros & {\tt clz rd, rs} \\
Bitwise AND & {\tt and rd, rs, rt} \\
Bitwise AND immediate & {\tt andi rd, rs, }$\mathtt{imm}_{16}$ \\
Bitwise NOR & {\tt nor rd, rs, rt} \\
Bitwise NOT & {\tt not rd, rs} \\
Bitwise OR & {\tt or rd, rs, rt} \\
Bitwise OR immediate & {\tt ori rd, rs, }$\mathtt{imm}_{16}$ \\
Bitwise XOR & {\tt xor rd, rs, rt} \\
Bitwise XOR immediate & {\tt xori rd, rs, }$\mathtt{imm}_{16}$ \\
Shift left (by immediate; fill with 0s) & {\tt sll rd, rs, }$\mathtt{imm}_5$ \\
Shift left (by register value; fill with 0s) & {\tt sllv rd, rs, rt} \\
Shift right (by immediate; fill with 0s) & {\tt srl rd, rs, }$\mathtt{imm}_5$ \\
Shift right (by register value; fill with 0s) & {\tt srlv rd, rs, rt} \\
Shift right arithmetic (by immediate; fill with MSB) & {\tt sra rd, rs, }$\mathtt{imm}_5$ \\
Shift right arithmetic (by register value; fill with MSB) & {\tt srav rd, rs, rt} \\
Circular shift left\pseudo & {\tt rol rd, rs, rt} \\
Circular shift right\pseudo & {\tt ror rd, rs, rt} \\
Load immediate\pseudo & {\tt li rd, }$\mathtt{imm}_{32}$ \\
System call (see table below) & {\tt syscall} 
\end{tabular}
\end{center}
The behavior of {\tt syscall} is controlled by register $\$v0$.
The following are useful codes:
\vspace*{-1em}
\begin{center}
\begin{tabular}{ll}
1 & Print int stored in $\$a0$ \\
%2 & Print float stored in $\$f12$ \\
%3 & Print double stored in $\$f12$ \\
4 & Print string whose address is stored in $\$a0$ \\
5 & Read int into $\$v0$ \\
%6 & Read float into $\$f0$ \\
%7 & Read double into $\$f0$ \\
8 & Read string into buffer at address $\$a0$ of length $\$a1$ \\
10 & Exit simulation \\
11 & Print character stored in $\$a0$ \\
12 & Read character into $\$v0$ 
\end{tabular}
\end{center}

\newpage
\noindent
\textbf{Comparison, branch, and jump instructions} \\
Set instructions place 1 or 0 in {\tt rd} depending on the
condition.
For example, {\tt seq} sets {\tt rd} to 1 when
{\tt rs} $=$ {\tt rt}.
\begin{center}
\begin{tabular}{ll}
Set if equal & {\tt seq rd, rs, rt} \\
Set if not equal & {\tt sne rd, rs, rt} \\
Set if greater than\pseudo & {\tt sgt rd, rs, rt} \\
Set if greater than or equal\pseudo & {\tt sge rd, rs, rt} \\
Set if greater than unsigned\pseudo & {\tt sgtu rd, rs, rt} \\
Set if greater than or equal unsigned\pseudo & {\tt sgeu rd, rs, rt} \\
Set if less than & {\tt slt rd, rs, rt} \\
Set if less than immediate & {\tt slti rt, rs, }$\mathtt{imm}_{16}$ \\
Set if less than or equal\pseudo & {\tt sle rd, rs, rt} \\
Set if less than unsigned & {\tt sltu rd, rs, rt} \\
Set if less than or equal unsigned\pseudo & {\tt sleu rd, rs, rt} \\
Set if less than unsigned immediate & {\tt sltiu rd, rs, }$\mathtt{imm}_{16}$ \\
Unconditional branch\pseudo & {\tt b label} \\
%Branch if equals zero\pseudo & {\tt beqz rs, label} \\
%Branch if does not equal zero\pseudo & {\tt bnez rs, label} \\
%Branch if greater than zero & {\tt bgtz rs, label} \\
%Branch if less than zero & {\tt bltz rs, label} \\
%Branch if greater than or equal to zero & {\tt bgez rs, label} \\
%Branch if less than or equal to zero & {\tt blez rs, label} \\
Branch if equal & {\tt beq rs, rt, label} \\
Branch if not equal & {\tt bne rs, rt, label} \\
Branch if greater than\pseudo & {\tt bgt rs, rt, label} \\
Branch if greater than or equal\pseudo & {\tt bge rs, rt, label} \\
Branch if greater than or equal unsigned\pseudo & {\tt bgeu rs, rt, label} \\
Branch if greater than unsigned\pseudo & {\tt bgtu rs, rt, label} \\
Branch if less than\pseudo & {\tt blt rs, rt, label} \\
Branch if less than or equal\pseudo & {\tt ble rs, rt, label} \\
Branch if less than or equal unsigned\pseudo & {\tt bleu rs, rt, label} \\
Branch if less than unsigned\pseudo & {\tt bltu rs, rt, label} \\
%Branch if greater than or equal to zero and link & {\tt bgezal rs, label} \\
%Branch if less than zero and link & {\tt bltzal rs, label} \\
%Jump immediate & {\tt j }$\mathtt{imm}_{26}$ \\
%Jump register & {\tt jr rs} \\
%Jump and link & {\tt jal }$\mathtt{imm}_{26}$ \\
%Jump to address in $s$ and link to $d$ & {\tt jalr rs, rd} \\
\end{tabular}
\end{center}

\noindent
\textbf{Memory instructions} \\
\begin{center}
\begin{tabular}{ll}
``the following goes in the text segment'' & {\tt .text} \\
``the following goes in the data segment'' & {\tt .data} \\
``store {\tt str} as a string'' & {\tt .ascii str} \\
``store {\tt str} as a null-terminated string'' & {\tt .asciiz str} \\
``store these words in memory'' & {\tt .word w1 w2 \ldots} \\
``reserve {\tt n} bytes of space'' & {\tt .space n} \\
``align on $2^n$-byte boundary'' & {\tt .align n} \\
Load address (not contents) \pseudo & {\tt la rd, address} \\
Load byte & {\tt lb rd, address} \\
Load unsigned byte & {\tt lbu rd, address} \\
Store byte & {\tt sb rt, address} \\
Load word & {\tt lw rd, address} \\
Store word & {\tt sw rd, address} \\
%Load halfword & {\tt lh rd, address} \\
%Load unsigned halfword & {\tt lhu rd, address} \\
\end{tabular}
\end{center}

The normal load instructions for less than a word sign-extend the
value; the unsigned versions do not.
Addresses for memory instructions can be any of the following forms:
\nopagebreak
\begin{center}
\begin{tabular}{ll}
(register) & contents of register \\
imm & immediate \\
imm(register) & contents of register + immediate \\
label & address of label \\
label $\pm$ imm & address of label $\pm$ immediate \\
label $\pm$ imm(register) & address of label $\pm$ (immediate+register)
\end{tabular}
\end{center}


%\subsection*{Floating point instructions}

\end{document}
