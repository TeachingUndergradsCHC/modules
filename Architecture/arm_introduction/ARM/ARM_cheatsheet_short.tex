\documentclass{article}

\usepackage{fullpage}
\usepackage{verbatim}
\usepackage{alltt}
\usepackage{url}

\pagestyle{empty}

\begin{document}

\begin{center}
{\LARGE {\bf ARM Assembly Operations}}
\end{center}

\vspace{1em}
\noindent
\textbf{Simplest Complete Program}
(compile with {\tt gcc -o filename filename.s}; run with {\tt ./filename})
\begin{alltt}
    .global main
    main:
        mov r7, \#1           @exit system call
        svc \#0
\end{alltt}

\vspace{1em}
\noindent
\textbf{Basic operations}

Argument {\tt dr} is the register in which to store the result.
Operands {\tt or} must be a register (e.g.\ {\tt r1}).
Operands {\tt oi} can be a register or an immediate (e.g.\ {\tt \#5}).
The argument {\tt \#0} for {\tt svc} must be this value.

\begin{tabular}{ll}
Add & {\tt add dr, or, oi} \\
Subtract ({\tt or} $-$ {\tt oi}) & {\tt sub dr, or, oi} \\
Reverse subtract ({\tt oi} $-$ {\tt or}) & {\tt rsb dr, or, oi} \\
Multiply ({\tt dr} and {\tt or1} cannot be the same) & {\tt mul dr, or1, or2} \\
Divide signed numbers\footnotemark ({\tt or1} $/$ {\tt or2}) & {\tt sdiv dr, or1, or2} \\
%Divide unsigned numbers ({\tt or1} by {\tt or2}) & {\tt udiv dr, or1, or2} \\
Logical shift left ({\tt oi} must be immediate) & {\tt lsl dr, or, oi} \\
Copy (from {\tt oi} to {\tt dr}) & {\tt mov dr, oi} \\
Compare {\tt or} to {\tt oi} and set comparison flags \hspace{2em} & {\tt cmp or, oi} \\
Branch to {\tt label} & {\tt b {\em address}} \\
Branch and link & {\tt bl {\em address}} \\ 
System call (see table below) & {\tt svc \#0} \\
\end{tabular} \\
\footnotetext{Requires the compile flag {\tt -mcpu=cortex-a7} for {\tt gcc};
  see \url{https://forums.raspberrypi.com/viewtopic.php?t=320122}}
\hspace*{6em}{\tt svc \#0} is controlled by the contents of register {\tt r7}: \\
\hspace*{6em}\begin{tabular}{ll}
1 & Exit program \\
3 & Read string ({\tt r2} bytes long) and store using address in {\tt r1}.
  {\tt r0} must be \#0 (standard input) \\
4 & Print string ({\tt r2} bytes long) whose address is stored in {\tt r1}.
  {\tt r0} must be \#1 (standard output) \\
\end{tabular}

\vspace{1em}
\noindent
\textbf{Conditional Suffixes}

All instructions can be used conditionally (based on the last call to
{\tt cmp}) by adding one of these suffixes.

\begin{tabular}{llrll}
If flags are set to ``equal'' & {\tt eq} & \hspace{1em} & If flags are set to ``not equal'' & {\tt ne} \\
If flags are set to ``greater than or equal'' & {\tt ge} & & If flags are set to ``less than or equal'' & {\tt le} \\
If flags are set to ``greater than'' & {\tt gt} & & If flags are set to ``less than'' & {\tt lt} \\
\end{tabular}

\vspace{1em}
\noindent
\textbf{Memory instructions}

\hspace*{2em}\begin{tabular}{ll}
Switch to the text segment & {\tt .text} \\
Switch to the data segment & {\tt .data} \\
Store {\tt str} as a null-terminated string & {\tt .asciz "str"} \\
Reserve {\tt oi} bytes of space ({\tt oi} must be immediate) \hspace{2em} & {\tt .space oi} \\
Create word ({\tt or} can be a string) & {\tt .word or} \\
Load word from {\tt \em address} & {\tt ldr dr, {\em address}} \\
Load address of {\tt \em labelText} & {\tt ldr dr, =\#{\em labelText}} \\
Store word at {\tt \em address} & {\tt str or, {\em address}} \\
Load byte from {\tt \em address} & {\tt ldrb dr, {\em address}} \\
Store byte at {\tt \em address} & {\tt strb or, {\em address}} \\
Push register values to the stack & {\tt push \{{\em reglist}\} } \\
Pop register values from the stack & {\tt pop \{{\em reglist}\}} \\
\end{tabular}

\vspace{0.5em}
{\tt\em address} can be {\tt [or]} or {\tt [or, oi]}, with the values
being added together when {\tt oi} is provided

{\tt\em reglist} is a comma-separated list of registers and ranges of
registers (e.g.\ {\tt r1, r5-r7})

one more line

\end{document}
