\documentclass{article}

\usepackage{fullpage}
\usepackage{verbatim}
\usepackage{alltt}
\usepackage{url}
\usepackage{amsmath}

\pagestyle{empty}

\begin{document}

\begin{center}
{\LARGE {\bf ARM Assembly Operations}}
\end{center}

\vspace{1em}
\noindent
\textbf{Simplest Complete Program}

Compile with {\tt gcc -o filename filename.s} and then run with {\tt ./filename}
\begin{alltt}
    .global main
    main:
        mov r7, \#1           @exit system call
        svc \#0
\end{alltt}

\vspace{1em}
\noindent
\textbf{Basic operations}

The right column gives the command followed by its arguments.
Argument {\tt dr} is the register in which to store the result.
Operands {\tt or} must be a register (e.g.\ {\tt r1}).
Operands {\tt oi} can be a register or an immediate (e.g.\ {\tt \#5}).
The argument {\tt \#0} for {\tt svc} must be this value.

\begin{tabular}{ll}
Add & {\tt add dr, or, oi} \\
Subtract ({\tt or} $-$ {\tt oi}) & {\tt sub dr, or, oi} \\
Reverse subtract ({\tt oi} $-$ {\tt or}) & {\tt rsb dr, or, oi} \\
Multiply ({\tt dr} and {\tt or1} cannot be the same) & {\tt mul dr, or1, or2} \\
%Multiply and add $\tt(or3 + (or1 * or2)$ & {\tt mla dr, or1, or2, or3} \\
%Multiply and subtract $\tt(or3 - (or1 * or2)$ & {\tt mls dr, or1, or2, or3} \\
Divide signed numbers\footnotemark ({\tt or1} $/$ {\tt or2}) & {\tt sdiv dr, or1, or2} \\
Divide unsigned numbers\footnotemark[1] ({\tt or1} $/$ {\tt or2}) & {\tt udiv dr, or1, or2} \\
Copy (from {\tt oi} to {\tt dr}) & {\tt mov dr, oi} \\
Compare {\tt or} to {\tt oi} and set comparison flags \hspace{2em} & {\tt cmp or, oi} \\
Branch to {\tt label} & {\tt b =label} \\
Branch and link & {\tt bl =label} \\
Return & {\tt ret} \\
System call (see table below) & {\tt svc \#0} \\
\end{tabular} \\
\footnotetext{Requires the compile flag {\tt -mcpu=cortex-a7} for {\tt gcc};
  see \url{https://forums.raspberrypi.com/viewtopic.php?t=320122}}
\hspace*{6em}{\tt svc \#0} is controlled by the contents of register {\tt r7}: \\
\hspace*{6em}\begin{tabular}{ll}
1 & Exit program \\
3 & Read string ({\tt r2} bytes long) and store using address in {\tt r1}.
  {\tt r0} must be \#0 (standard input) \\
4 & Print string ({\tt r2} bytes long) whose address is stored in {\tt r1}.
  {\tt r0} must be \#1 (standard output) \\
\end{tabular}

\vspace{1em}
\noindent
\textbf{Conditional Suffixes}

All instructions can be used conditionally (based on the last call to
{\tt cmp}) by adding one of these suffixes.

\begin{tabular}{ll}
If flags are set to ``equal'' & {\tt eq} \\
If flags are set to ``not equal'' & {\tt ne} \\
If flags are set to ``greater than or equal'' & {\tt ge} \\
If flags are set to ``less than or equal'' & {\tt le} \\
If flags are set to ``greater than'' & {\tt gt} \\
If flags are set to ``less than'' & {\tt lt} \\
Always execute (suffix has no effect) & {\tt al} \\
Never execute (creates a nop) & {\tt nv} \\
\end{tabular}

\vspace{1em}
\noindent
\textbf{Bitwise Instructions}

\begin{tabular}{ll}
Bitwise and & {\tt and dr, or, oi} \\
Bitwise or & {\tt orr dr, or, oi} \\
Bitwise exclusive or & {\tt eor dr, or, oi} \\
Bit clear & {\tt bic dr, or, oi} \\
Logical shift left ({\tt oi} must be immediate) & {\tt lsl dr, or, oi} \\
Logical shift right ({\tt oi} must be immediate) & {\tt lsr dr, or, oi} \\
Arithmetic shift right ({\tt oi} must be immediate) & {\tt asr dr, or, oi} \\
Rotate Right ({\tt oi} must be immediate) & {\tt ror dr, or, oi} \\
\end{tabular}

\pagebreak

\noindent
\textbf{Memory instructions}

\hspace*{2em}\begin{tabular}{ll}
Switch to the text segment & {\tt .text} \\
Switch to the data segment & {\tt .data} \\
Enter Thumb mode & {\tt .thumb} \\
Enter ARM mode & {\tt .arm} \\
Store {\tt str} as a null-terminated string & {\tt .asciz "str"} \\
Reserve {\tt oi} bytes of space ({\tt oi} must be immediate) \hspace{2em} & {\tt .space oi} \\
Create word ({\tt or} can be a string) & {\tt .word or} \\
Load word from {\tt \em address} & {\tt ldr dr, {\em address}} \\
Load address of {\tt \em labelText} & {\tt ldr dr, =\#{\em labelText}} \\
Store word at {\tt \em address} & {\tt str or, {\em address}} \\
Load byte from {\tt \em address} & {\tt ldrb dr, {\em address}} \\
Store byte at {\tt \em address} & {\tt strb or, {\em address}} \\
Push register values to the stack & {\tt push \{{\em reglist}\} } \\
Pop register values from the stack & {\tt pop \{{\em reglist}\}} \\
\end{tabular}

\vspace{0.5em}
{\tt\em address} can have any of the following forms:

\hspace*{4em}\begin{tabular}{ll}
address is stored in the register & {\tt [or]} \\
address is the sum of  register and the value of {\tt oi} & {\tt \verb-[or, oi]-} \\
address is stored in the register. Increment after loading/storing & {\tt \verb-[or]!-} \\
finds address of the label & {\tt ={\em labelText}} \\
\end{tabular}

{\tt\em reglist} can have any of the following forms:

\hspace*{4em}\begin{tabular}{ll}
single register & {\tt or} \\
range of registers & {\tt or1-or2} \\  
list of registers or ranges of registers & {\tt or1, or2-or3, or4} \\
\end{tabular}

\vspace{1em}
\noindent
\textbf{NEON Instructions}

Using NEON SIMD requires working with a different set of registers.
The registers {\tt d0}--{\tt d31} are double-word registers that store
64 bits each.
The registers {\tt q0}--{\tt q16} are quad-word registers that store
128 bits each.
Note that the {\tt d} and {\tt q} registers are different names for
the same data storage; {\tt d0} and {\tt d1} are the two halves of
{\tt q0} and so on.
Loading and storing into these registers requires special instructions:
\hspace*{2em}\begin{tabular}{ll}
Load into NEON  & {\tt vld1.{\em suf} \{\text{\tt reglist}\}, {\em address}} \\
Store from NEON into {\tt \em address} &  {\tt vst1.{\em suf} \{{\em reglist}\}, {\em address}} \\
\end{tabular} \\
where {\tt \em suf} is a suffix that is the number of bytes to be
loaded or stored and {\tt \em reglist} is a list of registers or
ranges of registers (using {\tt d} and {\tt q} registers).

\vspace{0.5em}
Once you have your data in the NEON registers, you can use standard
arithmetic and logic instructions on it with a few alterations: 
Add the prefix {\tt v} to the instruction, 
use either {\tt d} or {\tt q} registers as the parameters, 
and add a suffix to the end.
These suffixes give the size of each element (in bits) within the {\tt d} or
{\tt q} register: {\tt .8}, {\tt .16}, {\tt .32}, or {\tt .64}.
Optionally, you can also add {\tt u} or {\tt i} to specify that each
data element is unsigned or signed.
For example,
\begin{verbatim}
     vadd.u8 d0, d1, d2
\end{verbatim}
adds the four 8-bit unsigned integers stored in {\tt d1} and {\tt d2},
storing the result in {\tt d0}. 

To use NEON instructions, you must add the flag {\tt -mfpu=NEON} to the compilation command. 

\end{document}
